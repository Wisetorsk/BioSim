% Generated by Sphinx.
\def\sphinxdocclass{report}
\documentclass[a4paper,10pt,openany,oneside]{sphinxmanual}
\usepackage[utf8]{inputenc}
\DeclareUnicodeCharacter{00A0}{\nobreakspace}
\usepackage{cmap}
\usepackage[T1]{fontenc}
\usepackage[english]{babel}
\usepackage{times}
\usepackage[Bjarne]{fncychap}
\usepackage{longtable}
\usepackage{sphinx}
\usepackage{multirow}

\addto\captionsenglish{\renewcommand{\figurename}{Fig. }}
\addto\captionsenglish{\renewcommand{\tablename}{Table }}
\floatname{literal-block}{Listing }



\title{Biosim\_Project Documentation}
\date{January 20, 2016}
\release{1.1}
\author{Marius Skaug Kristiansen, Kristian Frafjord}
\newcommand{\sphinxlogo}{}
\renewcommand{\releasename}{Release}
\makeindex

\makeatletter
\def\PYG@reset{\let\PYG@it=\relax \let\PYG@bf=\relax%
    \let\PYG@ul=\relax \let\PYG@tc=\relax%
    \let\PYG@bc=\relax \let\PYG@ff=\relax}
\def\PYG@tok#1{\csname PYG@tok@#1\endcsname}
\def\PYG@toks#1+{\ifx\relax#1\empty\else%
    \PYG@tok{#1}\expandafter\PYG@toks\fi}
\def\PYG@do#1{\PYG@bc{\PYG@tc{\PYG@ul{%
    \PYG@it{\PYG@bf{\PYG@ff{#1}}}}}}}
\def\PYG#1#2{\PYG@reset\PYG@toks#1+\relax+\PYG@do{#2}}

\expandafter\def\csname PYG@tok@gd\endcsname{\def\PYG@tc##1{\textcolor[rgb]{0.63,0.00,0.00}{##1}}}
\expandafter\def\csname PYG@tok@gu\endcsname{\let\PYG@bf=\textbf\def\PYG@tc##1{\textcolor[rgb]{0.50,0.00,0.50}{##1}}}
\expandafter\def\csname PYG@tok@gt\endcsname{\def\PYG@tc##1{\textcolor[rgb]{0.00,0.27,0.87}{##1}}}
\expandafter\def\csname PYG@tok@gs\endcsname{\let\PYG@bf=\textbf}
\expandafter\def\csname PYG@tok@gr\endcsname{\def\PYG@tc##1{\textcolor[rgb]{1.00,0.00,0.00}{##1}}}
\expandafter\def\csname PYG@tok@cm\endcsname{\let\PYG@it=\textit\def\PYG@tc##1{\textcolor[rgb]{0.25,0.50,0.56}{##1}}}
\expandafter\def\csname PYG@tok@vg\endcsname{\def\PYG@tc##1{\textcolor[rgb]{0.73,0.38,0.84}{##1}}}
\expandafter\def\csname PYG@tok@m\endcsname{\def\PYG@tc##1{\textcolor[rgb]{0.13,0.50,0.31}{##1}}}
\expandafter\def\csname PYG@tok@mh\endcsname{\def\PYG@tc##1{\textcolor[rgb]{0.13,0.50,0.31}{##1}}}
\expandafter\def\csname PYG@tok@cs\endcsname{\def\PYG@tc##1{\textcolor[rgb]{0.25,0.50,0.56}{##1}}\def\PYG@bc##1{\setlength{\fboxsep}{0pt}\colorbox[rgb]{1.00,0.94,0.94}{\strut ##1}}}
\expandafter\def\csname PYG@tok@ge\endcsname{\let\PYG@it=\textit}
\expandafter\def\csname PYG@tok@vc\endcsname{\def\PYG@tc##1{\textcolor[rgb]{0.73,0.38,0.84}{##1}}}
\expandafter\def\csname PYG@tok@il\endcsname{\def\PYG@tc##1{\textcolor[rgb]{0.13,0.50,0.31}{##1}}}
\expandafter\def\csname PYG@tok@go\endcsname{\def\PYG@tc##1{\textcolor[rgb]{0.20,0.20,0.20}{##1}}}
\expandafter\def\csname PYG@tok@cp\endcsname{\def\PYG@tc##1{\textcolor[rgb]{0.00,0.44,0.13}{##1}}}
\expandafter\def\csname PYG@tok@gi\endcsname{\def\PYG@tc##1{\textcolor[rgb]{0.00,0.63,0.00}{##1}}}
\expandafter\def\csname PYG@tok@gh\endcsname{\let\PYG@bf=\textbf\def\PYG@tc##1{\textcolor[rgb]{0.00,0.00,0.50}{##1}}}
\expandafter\def\csname PYG@tok@ni\endcsname{\let\PYG@bf=\textbf\def\PYG@tc##1{\textcolor[rgb]{0.84,0.33,0.22}{##1}}}
\expandafter\def\csname PYG@tok@nl\endcsname{\let\PYG@bf=\textbf\def\PYG@tc##1{\textcolor[rgb]{0.00,0.13,0.44}{##1}}}
\expandafter\def\csname PYG@tok@nn\endcsname{\let\PYG@bf=\textbf\def\PYG@tc##1{\textcolor[rgb]{0.05,0.52,0.71}{##1}}}
\expandafter\def\csname PYG@tok@no\endcsname{\def\PYG@tc##1{\textcolor[rgb]{0.38,0.68,0.84}{##1}}}
\expandafter\def\csname PYG@tok@na\endcsname{\def\PYG@tc##1{\textcolor[rgb]{0.25,0.44,0.63}{##1}}}
\expandafter\def\csname PYG@tok@nb\endcsname{\def\PYG@tc##1{\textcolor[rgb]{0.00,0.44,0.13}{##1}}}
\expandafter\def\csname PYG@tok@nc\endcsname{\let\PYG@bf=\textbf\def\PYG@tc##1{\textcolor[rgb]{0.05,0.52,0.71}{##1}}}
\expandafter\def\csname PYG@tok@nd\endcsname{\let\PYG@bf=\textbf\def\PYG@tc##1{\textcolor[rgb]{0.33,0.33,0.33}{##1}}}
\expandafter\def\csname PYG@tok@ne\endcsname{\def\PYG@tc##1{\textcolor[rgb]{0.00,0.44,0.13}{##1}}}
\expandafter\def\csname PYG@tok@nf\endcsname{\def\PYG@tc##1{\textcolor[rgb]{0.02,0.16,0.49}{##1}}}
\expandafter\def\csname PYG@tok@si\endcsname{\let\PYG@it=\textit\def\PYG@tc##1{\textcolor[rgb]{0.44,0.63,0.82}{##1}}}
\expandafter\def\csname PYG@tok@s2\endcsname{\def\PYG@tc##1{\textcolor[rgb]{0.25,0.44,0.63}{##1}}}
\expandafter\def\csname PYG@tok@vi\endcsname{\def\PYG@tc##1{\textcolor[rgb]{0.73,0.38,0.84}{##1}}}
\expandafter\def\csname PYG@tok@nt\endcsname{\let\PYG@bf=\textbf\def\PYG@tc##1{\textcolor[rgb]{0.02,0.16,0.45}{##1}}}
\expandafter\def\csname PYG@tok@nv\endcsname{\def\PYG@tc##1{\textcolor[rgb]{0.73,0.38,0.84}{##1}}}
\expandafter\def\csname PYG@tok@s1\endcsname{\def\PYG@tc##1{\textcolor[rgb]{0.25,0.44,0.63}{##1}}}
\expandafter\def\csname PYG@tok@gp\endcsname{\let\PYG@bf=\textbf\def\PYG@tc##1{\textcolor[rgb]{0.78,0.36,0.04}{##1}}}
\expandafter\def\csname PYG@tok@sh\endcsname{\def\PYG@tc##1{\textcolor[rgb]{0.25,0.44,0.63}{##1}}}
\expandafter\def\csname PYG@tok@ow\endcsname{\let\PYG@bf=\textbf\def\PYG@tc##1{\textcolor[rgb]{0.00,0.44,0.13}{##1}}}
\expandafter\def\csname PYG@tok@sx\endcsname{\def\PYG@tc##1{\textcolor[rgb]{0.78,0.36,0.04}{##1}}}
\expandafter\def\csname PYG@tok@bp\endcsname{\def\PYG@tc##1{\textcolor[rgb]{0.00,0.44,0.13}{##1}}}
\expandafter\def\csname PYG@tok@c1\endcsname{\let\PYG@it=\textit\def\PYG@tc##1{\textcolor[rgb]{0.25,0.50,0.56}{##1}}}
\expandafter\def\csname PYG@tok@kc\endcsname{\let\PYG@bf=\textbf\def\PYG@tc##1{\textcolor[rgb]{0.00,0.44,0.13}{##1}}}
\expandafter\def\csname PYG@tok@c\endcsname{\let\PYG@it=\textit\def\PYG@tc##1{\textcolor[rgb]{0.25,0.50,0.56}{##1}}}
\expandafter\def\csname PYG@tok@mf\endcsname{\def\PYG@tc##1{\textcolor[rgb]{0.13,0.50,0.31}{##1}}}
\expandafter\def\csname PYG@tok@err\endcsname{\def\PYG@bc##1{\setlength{\fboxsep}{0pt}\fcolorbox[rgb]{1.00,0.00,0.00}{1,1,1}{\strut ##1}}}
\expandafter\def\csname PYG@tok@mb\endcsname{\def\PYG@tc##1{\textcolor[rgb]{0.13,0.50,0.31}{##1}}}
\expandafter\def\csname PYG@tok@ss\endcsname{\def\PYG@tc##1{\textcolor[rgb]{0.32,0.47,0.09}{##1}}}
\expandafter\def\csname PYG@tok@sr\endcsname{\def\PYG@tc##1{\textcolor[rgb]{0.14,0.33,0.53}{##1}}}
\expandafter\def\csname PYG@tok@mo\endcsname{\def\PYG@tc##1{\textcolor[rgb]{0.13,0.50,0.31}{##1}}}
\expandafter\def\csname PYG@tok@kd\endcsname{\let\PYG@bf=\textbf\def\PYG@tc##1{\textcolor[rgb]{0.00,0.44,0.13}{##1}}}
\expandafter\def\csname PYG@tok@mi\endcsname{\def\PYG@tc##1{\textcolor[rgb]{0.13,0.50,0.31}{##1}}}
\expandafter\def\csname PYG@tok@kn\endcsname{\let\PYG@bf=\textbf\def\PYG@tc##1{\textcolor[rgb]{0.00,0.44,0.13}{##1}}}
\expandafter\def\csname PYG@tok@o\endcsname{\def\PYG@tc##1{\textcolor[rgb]{0.40,0.40,0.40}{##1}}}
\expandafter\def\csname PYG@tok@kr\endcsname{\let\PYG@bf=\textbf\def\PYG@tc##1{\textcolor[rgb]{0.00,0.44,0.13}{##1}}}
\expandafter\def\csname PYG@tok@s\endcsname{\def\PYG@tc##1{\textcolor[rgb]{0.25,0.44,0.63}{##1}}}
\expandafter\def\csname PYG@tok@kp\endcsname{\def\PYG@tc##1{\textcolor[rgb]{0.00,0.44,0.13}{##1}}}
\expandafter\def\csname PYG@tok@w\endcsname{\def\PYG@tc##1{\textcolor[rgb]{0.73,0.73,0.73}{##1}}}
\expandafter\def\csname PYG@tok@kt\endcsname{\def\PYG@tc##1{\textcolor[rgb]{0.56,0.13,0.00}{##1}}}
\expandafter\def\csname PYG@tok@sc\endcsname{\def\PYG@tc##1{\textcolor[rgb]{0.25,0.44,0.63}{##1}}}
\expandafter\def\csname PYG@tok@sb\endcsname{\def\PYG@tc##1{\textcolor[rgb]{0.25,0.44,0.63}{##1}}}
\expandafter\def\csname PYG@tok@k\endcsname{\let\PYG@bf=\textbf\def\PYG@tc##1{\textcolor[rgb]{0.00,0.44,0.13}{##1}}}
\expandafter\def\csname PYG@tok@se\endcsname{\let\PYG@bf=\textbf\def\PYG@tc##1{\textcolor[rgb]{0.25,0.44,0.63}{##1}}}
\expandafter\def\csname PYG@tok@sd\endcsname{\let\PYG@it=\textit\def\PYG@tc##1{\textcolor[rgb]{0.25,0.44,0.63}{##1}}}

\def\PYGZbs{\char`\\}
\def\PYGZus{\char`\_}
\def\PYGZob{\char`\{}
\def\PYGZcb{\char`\}}
\def\PYGZca{\char`\^}
\def\PYGZam{\char`\&}
\def\PYGZlt{\char`\<}
\def\PYGZgt{\char`\>}
\def\PYGZsh{\char`\#}
\def\PYGZpc{\char`\%}
\def\PYGZdl{\char`\$}
\def\PYGZhy{\char`\-}
\def\PYGZsq{\char`\'}
\def\PYGZdq{\char`\"}
\def\PYGZti{\char`\~}
% for compatibility with earlier versions
\def\PYGZat{@}
\def\PYGZlb{[}
\def\PYGZrb{]}
\makeatother

\renewcommand\PYGZsq{\textquotesingle}

\begin{document}

\maketitle
\tableofcontents
\phantomsection\label{index::doc}

\begin{description}
\item[{This is a simulation of animals of two species with}] \leavevmode\begin{itemize}
\item {} 
a modifiable island

\item {} 
two species:

\end{itemize}
\begin{itemize}
\item {} 
Carnivores

\item {} 
Herbivores

\end{itemize}
\begin{itemize}
\item {} 
customizable parameters for island biomes and animals

\end{itemize}

\end{description}


\chapter{Island}
\label{island::doc}\label{island:welcome-to-biosim-project-s-documentation}\label{island:island}

\section{The island module}
\label{island:module-biosim.island}\label{island:the-island-module}\index{biosim.island (module)}\index{Island (class in biosim.island)}

\begin{fulllineitems}
\phantomsection\label{island:biosim.island.Island}\pysiglinewithargsret{\strong{class }\code{biosim.island.}\bfcode{Island}}{\emph{island\_map}}{}~\begin{description}
\item[{Class object: Island.}] \leavevmode\begin{itemize}
\item {} 
\textless{}var\textgreater{} = Island(String of landscape types)

\item {} 
\textless{}var\textgreater{}.build\_map() must run before simulation starts

\end{itemize}

\end{description}
\begin{quote}\begin{description}
\item[{Parameters}] \leavevmode
\textbf{\texttt{island\_map}} -- Map of island as string of ``biomes''

\end{description}\end{quote}
\index{aging() (biosim.island.Island method)}

\begin{fulllineitems}
\phantomsection\label{island:biosim.island.Island.aging}\pysiglinewithargsret{\bfcode{aging}}{}{}
Runs ageing method in each cell

\end{fulllineitems}

\index{build\_map() (biosim.island.Island method)}

\begin{fulllineitems}
\phantomsection\label{island:biosim.island.Island.build_map}\pysiglinewithargsret{\bfcode{build\_map}}{}{}
Builds island based on input string.
Island biomes are placed in a two dimensional array

\end{fulllineitems}

\index{death() (biosim.island.Island method)}

\begin{fulllineitems}
\phantomsection\label{island:biosim.island.Island.death}\pysiglinewithargsret{\bfcode{death}}{}{}
Runs death function in each cell

\end{fulllineitems}

\index{feeding() (biosim.island.Island method)}

\begin{fulllineitems}
\phantomsection\label{island:biosim.island.Island.feeding}\pysiglinewithargsret{\bfcode{feeding}}{}{}
Runs animal level feeding method on each animal in each cell

\end{fulllineitems}

\index{grow() (biosim.island.Island method)}

\begin{fulllineitems}
\phantomsection\label{island:biosim.island.Island.grow}\pysiglinewithargsret{\bfcode{grow}}{}{}
Runs through growth cycle for each cell in \textless{}var\textgreater{}.island.
Runs only if the cell i capable of growing food

\end{fulllineitems}

\index{individuals() (biosim.island.Island method)}

\begin{fulllineitems}
\phantomsection\label{island:biosim.island.Island.individuals}\pysiglinewithargsret{\bfcode{individuals}}{}{}
Returns the number of carnivores and herbivores in each cell
as array with each cell containing a dict over herbivores and carnivores

\end{fulllineitems}

\index{loss\_of\_weight() (biosim.island.Island method)}

\begin{fulllineitems}
\phantomsection\label{island:biosim.island.Island.loss_of_weight}\pysiglinewithargsret{\bfcode{loss\_of\_weight}}{}{}
Runs weightloss method in each cell

\end{fulllineitems}

\index{migration() (biosim.island.Island method)}

\begin{fulllineitems}
\phantomsection\label{island:biosim.island.Island.migration}\pysiglinewithargsret{\bfcode{migration}}{}{}
Runs the migration process for all cells. Starts at first cell in
shuffled list. Resets the has\_moved attribute of the animals when
all animals have moved.

\end{fulllineitems}

\index{one\_year() (biosim.island.Island method)}

\begin{fulllineitems}
\phantomsection\label{island:biosim.island.Island.one_year}\pysiglinewithargsret{\bfcode{one\_year}}{}{}
Simulates one year progression and returns array containing the
numbers of each species in each cell.

\end{fulllineitems}

\index{procreation() (biosim.island.Island method)}

\begin{fulllineitems}
\phantomsection\label{island:biosim.island.Island.procreation}\pysiglinewithargsret{\bfcode{procreation}}{}{}
Runs animal level breeding method on each animal in each cell

\end{fulllineitems}

\index{shuffle\_coordinates() (biosim.island.Island method)}

\begin{fulllineitems}
\phantomsection\label{island:biosim.island.Island.shuffle_coordinates}\pysiglinewithargsret{\bfcode{shuffle\_coordinates}}{}{}
Makes a shuffled list of the coordinates for all the cells on map.
\begin{quote}\begin{description}
\item[{Returns}] \leavevmode
Shuffled list of coordinates

\end{description}\end{quote}

\end{fulllineitems}

\index{surrounding\_cells() (biosim.island.Island method)}

\begin{fulllineitems}
\phantomsection\label{island:biosim.island.Island.surrounding_cells}\pysiglinewithargsret{\bfcode{surrounding\_cells}}{\emph{coordinate}}{}
Calculates surrounding cells that animals can migrate to.
If the neighboring cell is either mountain or ocean it does not
append the selected coordinates
\begin{quote}\begin{description}
\item[{Parameters}] \leavevmode
\textbf{\texttt{coordinate}} -- A given coordinate

\item[{Returns}] \leavevmode
List of surrounding cells which animals can migrate to.

\end{description}\end{quote}

\end{fulllineitems}

\index{surrounding\_cells\_relative\_food() (biosim.island.Island method)}

\begin{fulllineitems}
\phantomsection\label{island:biosim.island.Island.surrounding_cells_relative_food}\pysiglinewithargsret{\bfcode{surrounding\_cells\_relative\_food}}{\emph{y}, \emph{x}, \emph{species}}{}
Makes nested lists for the coordinates of the surrounding cells and
their amount of relative food.
\begin{quote}\begin{description}
\item[{Parameters}] \leavevmode\begin{itemize}
\item {} 
\textbf{\texttt{y}} -- y coordinate for the center cell

\item {} 
\textbf{\texttt{x}} -- x coordinate for the center cell

\item {} 
\textbf{\texttt{species}} -- String with the name of the specie to calculate relative

\end{itemize}

\end{description}\end{quote}

food for.
:return: nested list with surrounding cell and their calculated relative
food

\end{fulllineitems}


\end{fulllineitems}



\chapter{Landscape}
\label{landscape::doc}\label{landscape:landscape}

\section{The landscape module}
\label{landscape:module-biosim.landscape}\label{landscape:the-landscape-module}\index{biosim.landscape (module)}\index{Desert (class in biosim.landscape)}

\begin{fulllineitems}
\phantomsection\label{landscape:biosim.landscape.Desert}\pysiglinewithargsret{\strong{class }\code{biosim.landscape.}\bfcode{Desert}}{\emph{carnivores=None}, \emph{herbivores=None}}{}
Landscape subclass Desert
Inhabitable for herbivores, but carnivores can feed on herbivores in desert

\end{fulllineitems}

\index{Jungle (class in biosim.landscape)}

\begin{fulllineitems}
\phantomsection\label{landscape:biosim.landscape.Jungle}\pysiglinewithargsret{\strong{class }\code{biosim.landscape.}\bfcode{Jungle}}{\emph{carnivores=None}, \emph{herbivores=None}}{}
Landscape subclass Jungle.
Habitable and food is replenished to maximum level each year
\index{grow\_food() (biosim.landscape.Jungle method)}

\begin{fulllineitems}
\phantomsection\label{landscape:biosim.landscape.Jungle.grow_food}\pysiglinewithargsret{\bfcode{grow\_food}}{}{}
Replenishes the amount of food in the jungle cell to f\_max

\end{fulllineitems}


\end{fulllineitems}

\index{Landscape (class in biosim.landscape)}

\begin{fulllineitems}
\phantomsection\label{landscape:biosim.landscape.Landscape}\pysiglinewithargsret{\strong{class }\code{biosim.landscape.}\bfcode{Landscape}}{\emph{carnivores=None}, \emph{herbivores=None}}{}
Superclass Landscape

Constructor for Landscape.
\begin{quote}\begin{description}
\item[{Parameters}] \leavevmode
\textbf{\texttt{carnivores}} -- Instances of carnivores as list of

\end{description}\end{quote}

``Carnivore()'' instances
:param herbivores: Instances of herbivores as list of
``Herbivore()'' instances
\index{age\_cycle() (biosim.landscape.Landscape method)}

\begin{fulllineitems}
\phantomsection\label{landscape:biosim.landscape.Landscape.age_cycle}\pysiglinewithargsret{\bfcode{age\_cycle}}{}{}
Each animals age is incremented by one year

\end{fulllineitems}

\index{avg\_age() (biosim.landscape.Landscape method)}

\begin{fulllineitems}
\phantomsection\label{landscape:biosim.landscape.Landscape.avg_age}\pysiglinewithargsret{\bfcode{avg\_age}}{}{}
Returns the average age of population
\begin{quote}\begin{description}
\item[{Returns}] \leavevmode
(``herbivores age'', ``carnivores age'')

\end{description}\end{quote}

\end{fulllineitems}

\index{avg\_fitness() (biosim.landscape.Landscape method)}

\begin{fulllineitems}
\phantomsection\label{landscape:biosim.landscape.Landscape.avg_fitness}\pysiglinewithargsret{\bfcode{avg\_fitness}}{}{}
Method used for testing
Returns the average fitness of the population
\begin{quote}\begin{description}
\item[{Returns}] \leavevmode
(``herbivores fitness'', ``carnivores fitness'')

\end{description}\end{quote}

\end{fulllineitems}

\index{breeding\_cycle() (biosim.landscape.Landscape method)}

\begin{fulllineitems}
\phantomsection\label{landscape:biosim.landscape.Landscape.breeding_cycle}\pysiglinewithargsret{\bfcode{breeding\_cycle}}{}{}
Starts the breeding cycle for both species in a single cell
If breeding is successful, the method appends a new animal
of the same species to the list of animals

\end{fulllineitems}

\index{calc\_fitness() (biosim.landscape.Landscape static method)}

\begin{fulllineitems}
\phantomsection\label{landscape:biosim.landscape.Landscape.calc_fitness}\pysiglinewithargsret{\strong{static }\bfcode{calc\_fitness}}{\emph{animals}}{}
Makes a sorted list for animal fitness of the input list of animal
instances. Highest fitness first.
\begin{quote}\begin{description}
\item[{Parameters}] \leavevmode
\textbf{\texttt{animals}} -- List of animal instances

\item[{Returns}] \leavevmode
Sorted list of animal instances with fitness values in a

\end{description}\end{quote}

tuple consisting of (\textless{}class instance\textgreater{}, ``fitness value'')

\end{fulllineitems}

\index{death\_cycle() (biosim.landscape.Landscape method)}

\begin{fulllineitems}
\phantomsection\label{landscape:biosim.landscape.Landscape.death_cycle}\pysiglinewithargsret{\bfcode{death\_cycle}}{}{}
Starts the death-function for each animal.
Removes animals who are ``dead'' (Animal death method returns ``True'')

\end{fulllineitems}

\index{feeding\_cycle() (biosim.landscape.Landscape method)}

\begin{fulllineitems}
\phantomsection\label{landscape:biosim.landscape.Landscape.feeding_cycle}\pysiglinewithargsret{\bfcode{feeding\_cycle}}{}{}
Starts the feeding cycle for herbivores in a single cell.
Highest fitness first.

\end{fulllineitems}

\index{herbivore\_weight() (biosim.landscape.Landscape method)}

\begin{fulllineitems}
\phantomsection\label{landscape:biosim.landscape.Landscape.herbivore_weight}\pysiglinewithargsret{\bfcode{herbivore\_weight}}{}{}
Updates available food in cell for carnivore based on
the total weight of herbivores in cell

\end{fulllineitems}

\index{migration\_cycle\_carn() (biosim.landscape.Landscape method)}

\begin{fulllineitems}
\phantomsection\label{landscape:biosim.landscape.Landscape.migration_cycle_carn}\pysiglinewithargsret{\bfcode{migration\_cycle\_carn}}{\emph{\_list}}{}
Starts the migration cycle for the carnivores in the cell.
\begin{quote}\begin{description}
\item[{Parameters}] \leavevmode
\textbf{\texttt{\_list}} -- Nested list with possible coordinates the carnivores may

\end{description}\end{quote}

move to and the relative food for each cell.
\begin{quote}\begin{description}
\item[{Returns}] \leavevmode
List of animals that are migrating and their new position

\end{description}\end{quote}

\end{fulllineitems}

\index{migration\_cycle\_herb() (biosim.landscape.Landscape method)}

\begin{fulllineitems}
\phantomsection\label{landscape:biosim.landscape.Landscape.migration_cycle_herb}\pysiglinewithargsret{\bfcode{migration\_cycle\_herb}}{\emph{\_list}}{}
Starts the migration cycle for the herbivores in the cell.
\begin{quote}\begin{description}
\item[{Parameters}] \leavevmode
\textbf{\texttt{\_list}} -- Nested list with possible coordinates the herbivores may

\end{description}\end{quote}

move to and the relative food for each cell.
:return: List of animals that are migrating and their new position

\end{fulllineitems}

\index{number\_of\_individuals() (biosim.landscape.Landscape method)}

\begin{fulllineitems}
\phantomsection\label{landscape:biosim.landscape.Landscape.number_of_individuals}\pysiglinewithargsret{\bfcode{number\_of\_individuals}}{}{}
Returns number of individuals in cell.
\begin{quote}\begin{description}
\item[{Returns}] \leavevmode
dictionary containing number

\end{description}\end{quote}

of carnivores and number of herbivores

\end{fulllineitems}

\index{relative\_food\_carn() (biosim.landscape.Landscape method)}

\begin{fulllineitems}
\phantomsection\label{landscape:biosim.landscape.Landscape.relative_food_carn}\pysiglinewithargsret{\bfcode{relative\_food\_carn}}{}{}
Calculates the amount of relative food in cell for carnivores.
\begin{quote}\begin{description}
\item[{Returns}] \leavevmode
Amount of relative food for carnivores

\end{description}\end{quote}

\end{fulllineitems}

\index{relative\_food\_herb() (biosim.landscape.Landscape method)}

\begin{fulllineitems}
\phantomsection\label{landscape:biosim.landscape.Landscape.relative_food_herb}\pysiglinewithargsret{\bfcode{relative\_food\_herb}}{}{}
Calculates the amount of relative food in cell for herbivores.
\begin{quote}\begin{description}
\item[{Returns}] \leavevmode
Amount of relative food for herbivore

\end{description}\end{quote}

\end{fulllineitems}

\index{set\_parameters() (biosim.landscape.Landscape class method)}

\begin{fulllineitems}
\phantomsection\label{landscape:biosim.landscape.Landscape.set_parameters}\pysiglinewithargsret{\strong{classmethod }\bfcode{set\_parameters}}{\emph{new\_params}}{}
Updates parameters. Raises ValueError if values are invalid
\begin{quote}\begin{description}
\item[{Parameters}] \leavevmode
\textbf{\texttt{new\_params}} -- New set of parameters as dictionary

\end{description}\end{quote}

\end{fulllineitems}

\index{weightloss\_cycle() (biosim.landscape.Landscape method)}

\begin{fulllineitems}
\phantomsection\label{landscape:biosim.landscape.Landscape.weightloss_cycle}\pysiglinewithargsret{\bfcode{weightloss\_cycle}}{}{}
Each animal loses weight according to formula; ``eta'' * ``weight''

\end{fulllineitems}


\end{fulllineitems}

\index{Mountain (class in biosim.landscape)}

\begin{fulllineitems}
\phantomsection\label{landscape:biosim.landscape.Mountain}\pysiglinewithargsret{\strong{class }\code{biosim.landscape.}\bfcode{Mountain}}{\emph{carnivores=None}, \emph{herbivores=None}}{}
Landscape subclass Mountain
Impassable terrain for both species

\end{fulllineitems}

\index{Ocean (class in biosim.landscape)}

\begin{fulllineitems}
\phantomsection\label{landscape:biosim.landscape.Ocean}\pysiglinewithargsret{\strong{class }\code{biosim.landscape.}\bfcode{Ocean}}{\emph{carnivores=None}, \emph{herbivores=None}}{}
Landscape subclass Ocean
Impassable terrain for both species

\end{fulllineitems}

\index{Savannah (class in biosim.landscape)}

\begin{fulllineitems}
\phantomsection\label{landscape:biosim.landscape.Savannah}\pysiglinewithargsret{\strong{class }\code{biosim.landscape.}\bfcode{Savannah}}{\emph{carnivores=None}, \emph{herbivores=None}}{}
Landscape subclass Savannah.
Habitable, but food grows at a reduced rate.
\index{grow\_food() (biosim.landscape.Savannah method)}

\begin{fulllineitems}
\phantomsection\label{landscape:biosim.landscape.Savannah.grow_food}\pysiglinewithargsret{\bfcode{grow\_food}}{}{}
Replenishes the amount of food in Savannah cell according to formula

\end{fulllineitems}


\end{fulllineitems}



\chapter{Animals}
\label{animals:animals}\label{animals::doc}

\section{The animals module}
\label{animals:the-animals-module}\label{animals:module-biosim.animals}\index{biosim.animals (module)}\index{Animal (class in biosim.animals)}

\begin{fulllineitems}
\phantomsection\label{animals:biosim.animals.Animal}\pysiglinewithargsret{\strong{class }\code{biosim.animals.}\bfcode{Animal}}{\emph{weight=None}, \emph{age=0}, \emph{coordinates=(1}, \emph{1)}}{}
Superclass ``Animal'' for herbivores and carnivores
\begin{quote}\begin{description}
\item[{Parameters}] \leavevmode\begin{itemize}
\item {} 
\textbf{\texttt{weight}} -- Default None results in gaussian distribution of weight

\item {} 
\textbf{\texttt{age}} -- The starting age of animal

\item {} 
\textbf{\texttt{coordinates}} -- Starting coordinates of the animal

\end{itemize}

\end{description}\end{quote}
\index{ageing() (biosim.animals.Animal method)}

\begin{fulllineitems}
\phantomsection\label{animals:biosim.animals.Animal.ageing}\pysiglinewithargsret{\bfcode{ageing}}{}{}
Increment age by one per year

\end{fulllineitems}

\index{breeding() (biosim.animals.Animal method)}

\begin{fulllineitems}
\phantomsection\label{animals:biosim.animals.Animal.breeding}\pysiglinewithargsret{\bfcode{breeding}}{\emph{individuals}}{}
Calculates if the animal will give birth based on animals present in
cell, weight of animal and set parameters
\begin{quote}\begin{description}
\item[{Parameters}] \leavevmode
\textbf{\texttt{individuals}} -- number of individuals in cell

\item[{Returns}] \leavevmode
Returns birth weight if it gives birth or None.

\end{description}\end{quote}

\end{fulllineitems}

\index{check\_migrate() (biosim.animals.Animal method)}

\begin{fulllineitems}
\phantomsection\label{animals:biosim.animals.Animal.check_migrate}\pysiglinewithargsret{\bfcode{check\_migrate}}{}{}
Check if the animal wants to migrate based on set parameters
\begin{quote}\begin{description}
\item[{Returns}] \leavevmode
True if animal will migrate

\end{description}\end{quote}

\end{fulllineitems}

\index{death() (biosim.animals.Animal method)}

\begin{fulllineitems}
\phantomsection\label{animals:biosim.animals.Animal.death}\pysiglinewithargsret{\bfcode{death}}{}{}
Calculates if the animal dies or not based on fitness and set parameters
\begin{quote}\begin{description}
\item[{Returns}] \leavevmode
True if the animal dies, False otherwise

\end{description}\end{quote}

\end{fulllineitems}

\index{migrate() (biosim.animals.Animal method)}

\begin{fulllineitems}
\phantomsection\label{animals:biosim.animals.Animal.migrate}\pysiglinewithargsret{\bfcode{migrate}}{\emph{\_list}}{}
Calculates if the herbivore will migrate and returns either the new
coordinates or the current coordinates.
\begin{quote}\begin{description}
\item[{Parameters}] \leavevmode
\textbf{\texttt{\_list}} -- Nested list of tuples with surrounding positions as first

\end{description}\end{quote}

element and relative food as second element.
:return: New coordinates for the animal if it migrates or the old
if it does not.

\end{fulllineitems}

\index{set\_parameters() (biosim.animals.Animal class method)}

\begin{fulllineitems}
\phantomsection\label{animals:biosim.animals.Animal.set_parameters}\pysiglinewithargsret{\strong{classmethod }\bfcode{set\_parameters}}{\emph{new\_params}}{}
Updates parameters. Raises ValueError if values are invalid
\begin{quote}\begin{description}
\item[{Parameters}] \leavevmode
\textbf{\texttt{new\_params}} -- New set of parameters as dictionary

\end{description}\end{quote}

\end{fulllineitems}

\index{update\_fitness() (biosim.animals.Animal method)}

\begin{fulllineitems}
\phantomsection\label{animals:biosim.animals.Animal.update_fitness}\pysiglinewithargsret{\bfcode{update\_fitness}}{}{}
Re-calculates the animal's fitness based on age and weight

\end{fulllineitems}

\index{weightloss() (biosim.animals.Animal method)}

\begin{fulllineitems}
\phantomsection\label{animals:biosim.animals.Animal.weightloss}\pysiglinewithargsret{\bfcode{weightloss}}{}{}
Recalculates the animals weight according to ``eta'' and original weight

\end{fulllineitems}


\end{fulllineitems}

\index{Carnivore (class in biosim.animals)}

\begin{fulllineitems}
\phantomsection\label{animals:biosim.animals.Carnivore}\pysiglinewithargsret{\strong{class }\code{biosim.animals.}\bfcode{Carnivore}}{\emph{weight=None}, \emph{age=0}, \emph{coordinates=(1}, \emph{1)}}{}
Animal subclass Carnivore
\index{feeding() (biosim.animals.Carnivore method)}

\begin{fulllineitems}
\phantomsection\label{animals:biosim.animals.Carnivore.feeding}\pysiglinewithargsret{\bfcode{feeding}}{\emph{herbivores}}{}
Calculate if the carnivore will feed based on its own fitness and
the fitness of the herbivore, and gain weight. Removes eaten
herbivores.
\begin{quote}\begin{description}
\item[{Parameters}] \leavevmode
\textbf{\texttt{herbivores}} -- List of herbivores in cell

\item[{Returns}] \leavevmode
Updated list of herbivores in cell after eating

\end{description}\end{quote}

\end{fulllineitems}


\end{fulllineitems}

\index{Herbivore (class in biosim.animals)}

\begin{fulllineitems}
\phantomsection\label{animals:biosim.animals.Herbivore}\pysiglinewithargsret{\strong{class }\code{biosim.animals.}\bfcode{Herbivore}}{\emph{weight=None}, \emph{age=0}, \emph{coordinates=(1}, \emph{1)}}{}
Animal subclass Herbivore.
\index{feeding() (biosim.animals.Herbivore method)}

\begin{fulllineitems}
\phantomsection\label{animals:biosim.animals.Herbivore.feeding}\pysiglinewithargsret{\bfcode{feeding}}{\emph{available\_food}}{}
Herbivore feeding method. The animal will feed based on amount of
available food in cell, and returns the result. If the amount left in
cell is less than the animals ``hunger'', the animal will eat the
remaining amount and returns 0
\begin{quote}\begin{description}
\item[{Parameters}] \leavevmode
\textbf{\texttt{available\_food}} -- available food before eating

\item[{Returns}] \leavevmode
new amount of food left after eating

\end{description}\end{quote}

\end{fulllineitems}


\end{fulllineitems}



\chapter{Simulation}
\label{simulation::doc}\label{simulation:simulation}

\section{The simulation module}
\label{simulation:module-biosim.simulation}\label{simulation:the-simulation-module}\index{biosim.simulation (module)}\index{BioSim (class in biosim.simulation)}

\begin{fulllineitems}
\phantomsection\label{simulation:biosim.simulation.BioSim}\pysiglinewithargsret{\strong{class }\code{biosim.simulation.}\bfcode{BioSim}}{\emph{island\_map=None}, \emph{ini\_pop=None}, \emph{seed=None}}{}
Class BioSim
Main simulation class for biosim project
\begin{description}
\item[{Constructor creates island from given map, and adds}] \leavevmode
initial population to island.

\item[{All parameters needed to create animals are contained within}] \leavevmode
ini\_pop.

\end{description}
\begin{quote}\begin{description}
\item[{Parameters}] \leavevmode
\textbf{\texttt{island\_map}} -- String containing letters representing each

\end{description}\end{quote}

type of landscape
:param ini\_pop: List of initial\_populations. {[}pop1, pop2{]}
:param seed: Seed for rng
\index{add\_population() (biosim.simulation.BioSim method)}

\begin{fulllineitems}
\phantomsection\label{simulation:biosim.simulation.BioSim.add_population}\pysiglinewithargsret{\bfcode{add\_population}}{\emph{population}}{}
Adds given population to cells. Location stored in given population
:param population: list of populations

\end{fulllineitems}

\index{animal() (biosim.simulation.BioSim method)}

\begin{fulllineitems}
\phantomsection\label{simulation:biosim.simulation.BioSim.animal}\pysiglinewithargsret{\bfcode{animal}}{}{}
Returns total number of animals per type
:return anim: Dict of animals per type

\end{fulllineitems}

\index{animals\_by\_species() (biosim.simulation.BioSim method)}

\begin{fulllineitems}
\phantomsection\label{simulation:biosim.simulation.BioSim.animals_by_species}\pysiglinewithargsret{\bfcode{animals\_by\_species}}{}{}
Prints number of Animals per type on island

\end{fulllineitems}

\index{heatmap() (biosim.simulation.BioSim static method)}

\begin{fulllineitems}
\phantomsection\label{simulation:biosim.simulation.BioSim.heatmap}\pysiglinewithargsret{\strong{static }\bfcode{heatmap}}{\emph{island\_results}}{}
Returns two nested lists; {[}row{[}cell{]}{]}, one for herbivore population
and one for carnivore population.
:param island\_results: Array containing population data for one year
:return kart\_herb, kart\_carn: Map over populations

\end{fulllineitems}

\index{per\_cell\_animal\_count() (biosim.simulation.BioSim method)}

\begin{fulllineitems}
\phantomsection\label{simulation:biosim.simulation.BioSim.per_cell_animal_count}\pysiglinewithargsret{\bfcode{per\_cell\_animal\_count}}{}{}
Prints the total amount of animals in island

\end{fulllineitems}

\index{plot\_update() (biosim.simulation.BioSim method)}

\begin{fulllineitems}
\phantomsection\label{simulation:biosim.simulation.BioSim.plot_update}\pysiglinewithargsret{\bfcode{plot\_update}}{\emph{years}, \emph{abscissa}, \emph{ordinate}, \emph{colour\_herb}, \emph{colour\_carn}}{}
Updates the plot n\_steps years
:param years: Number of years to plot
:param abscissa: Xrange
:param ordinate: Yrange
:param colour\_herb: Colour of heatmap (default None)
:param colour\_carn: Colour of heatmap (default None)

\end{fulllineitems}

\index{simulate() (biosim.simulation.BioSim method)}

\begin{fulllineitems}
\phantomsection\label{simulation:biosim.simulation.BioSim.simulate}\pysiglinewithargsret{\bfcode{simulate}}{\emph{num\_steps=100}, \emph{vis\_steps=1}, \emph{img\_steps=2000}, \emph{abscissa=None}, \emph{ordinate=20000}, \emph{colour\_herb=None}, \emph{colour\_carn=None}}{}
Runs simulation num\_steps number of years
\begin{quote}\begin{description}
\item[{Parameters}] \leavevmode\begin{itemize}
\item {} 
\textbf{\texttt{num\_steps}} -- Number of years to simulate

\item {} 
\textbf{\texttt{vis\_steps}} -- Number of years between each time results are drawn

\item {} 
\textbf{\texttt{img\_steps}} -- Number of years between each .png file created

\item {} 
\textbf{\texttt{abscissa}} -- Length of x-axis

\item {} 
\textbf{\texttt{ordinate}} -- Length of y-axis

\item {} 
\textbf{\texttt{colour\_herb}} -- Colour for chart of herbivore population

\item {} 
\textbf{\texttt{colour\_carn}} -- Colour for chart of herbivore population

\end{itemize}

\end{description}\end{quote}

\end{fulllineitems}

\index{total\_number\_of\_animals() (biosim.simulation.BioSim method)}

\begin{fulllineitems}
\phantomsection\label{simulation:biosim.simulation.BioSim.total_number_of_animals}\pysiglinewithargsret{\bfcode{total\_number\_of\_animals}}{}{}
Prints the total number of animals on island

\end{fulllineitems}

\index{years\_simulated() (biosim.simulation.BioSim method)}

\begin{fulllineitems}
\phantomsection\label{simulation:biosim.simulation.BioSim.years_simulated}\pysiglinewithargsret{\bfcode{years\_simulated}}{}{}
Prints the total number of years simulated

\end{fulllineitems}


\end{fulllineitems}



\chapter{Indices and tables}
\label{index:indices-and-tables}\begin{itemize}
\item {} 
\DUspan{xref,std,std-ref}{genindex}

\item {} 
\DUspan{xref,std,std-ref}{modindex}

\item {} 
\DUspan{xref,std,std-ref}{search}

\end{itemize}


\renewcommand{\indexname}{Python Module Index}
\begin{theindex}
\def\bigletter#1{{\Large\sffamily#1}\nopagebreak\vspace{1mm}}
\bigletter{b}
\item {\texttt{biosim.animals}}, \pageref{animals:module-biosim.animals}
\item {\texttt{biosim.island}}, \pageref{island:module-biosim.island}
\item {\texttt{biosim.landscape}}, \pageref{landscape:module-biosim.landscape}
\item {\texttt{biosim.simulation}}, \pageref{simulation:module-biosim.simulation}
\end{theindex}

\renewcommand{\indexname}{Index}
\printindex
\end{document}
